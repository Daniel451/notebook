\section{Git}

	\subsection{Befehlsübersicht}

		\renewcommand{\arraystretch}{1.5}
		\begin{longtable}{|p{7cm}|p{8cm}|}
		  
		  	\hline

		  	\rowcolor{tableheadcolor}
		  	\textbf{Befehl} & \textbf{Auswirkung}

		  	\\ \hline

		  	%%%%%%%%%%%%%%%%%%%%%%%
		  	% git add
		  	%%%%%%%%%%%%%%%%%%%%%%%
		  	\rowcolor{tablerowcolor_cyan}
		  	git add & Fügt eine Datei der Stage hinzu
		  	\\ \hline
		  	\rowcolor{tablerowcolor_cyan}
		  	git add *.txt & Fügt alle txt-Dateien der Stage hinzu
		  	\\ \hline
		  	\rowcolor{tablerowcolor_cyan}
		  	git add . & Fügt alle Dateien (die unstaged sind) der Stage hinzu
		  	\\ \hline

		  	%%%%%%%%%%%%%%%%%%%%%%%
		  	% git branch
		  	%%%%%%%%%%%%%%%%%%%%%%%
		  	\rowcolor{tablerowcolor_orange}
		  	git branch & Listet alle vornandenen branches auf
		  	\\ \hline
		  	\rowcolor{tablerowcolor_orange}
		  	git branch -a & Listet alle remote und local branches auf
		  	\\ \hline
		  	\rowcolor{tablerowcolor_orange}
		  	git branch -r & Listet alle remote branches auf
		  	\\ \hline
		  	\rowcolor{tablerowcolor_orange}
		  	git branch feature & Erzeugt den neuen branch feature
		  	\\ \hline
		  	\rowcolor{tablerowcolor_orange}
		  	git branch -d feature & Löscht den branch feature
		  	\\ \hline
		  	\rowcolor{tablerowcolor_orange}
		  	git branch -u origin/feature & Weist den aktuellen branch an, den remote branch origin/feature zu tracken
		  	\\ \hline


		  	%%%%%%%%%%%%%%%%%%%%%%%
		  	% git clone
		  	%%%%%%%%%%%%%%%%%%%%%%%
		  	\rowcolor{tablerowcolor_cyan}
		  	git clone git@github.com:user/repository.git & Klont das remote repository in das aktuelle Verzeichnis
		  	\\ \hline
		  	\rowcolor{tablerowcolor_cyan}
		  	git clone git@github.com:user/repository.git zielverzeichnis & Klont das remote repository in das Zielverzeichnis
		  	\\ \hline


		  	%%%%%%%%%%%%%%%%%%%%%%%
		  	% git commit
		  	%%%%%%%%%%%%%%%%%%%%%%%
		  	\rowcolor{tablerowcolor_orange}
		  	git commit & Erzeugt einen neuen commit mit allen Dateien, die aktuell gestaged sind
		  	\\ \hline
		  	\rowcolor{tablerowcolor_orange}
		  	git commit --amend & Korrigiert das letzte commit (Korrigierung der commit-message plus hinzufügen der aktuell gestageten Dateien)
		  	\\ \hline


		  	%%%%%%%%%%%%%%%%%%%%%%%
		  	% git checkout
		  	%%%%%%%%%%%%%%%%%%%%%%%
		  	\rowcolor{tablerowcolor_cyan}
		  	git checkout master & Wechselt auf den branch master
		  	\\ \hline
		  	\rowcolor{tablerowcolor_cyan}
		  	git checkout origin/experimental & Wechselt in den origin branch origin/experimental (nur zum anschauen, nicht zum editieren, dann vorher immer erst eine local copy vom origin branch erstellen)
		  	\\ \hline
		  	\rowcolor{tablerowcolor_cyan}
		  	git checkout -- file.txt & Setzt file.txt auf den Stand des letzten commits (aktueller branch) zurück
		  	\\ \hline
		  	\rowcolor{tablerowcolor_cyan}
		  	git checkout -b local\_test origin/remote\_test & Erstellt den neuen branch local\_test im local repository und linked auf remote\_test für zukünftige fetches\/pulls
		  	\\ \hline


		  	%%%%%%%%%%%%%%%%%%%%%%%
		  	% git diff
		  	%%%%%%%%%%%%%%%%%%%%%%%
		  	\rowcolor{tablerowcolor_orange}
		  	git diff & Zeigt alle Änderungen seit dem letzten commit, die noch nicht gestaged sind
		  	\\ \hline
		  	\rowcolor{tablerowcolor_orange}
		  	git diff --staged & Zeigt alle Änderungen seit dem letzten commit, die gestaged sind
		  	\\ \hline


		  	%%%%%%%%%%%%%%%%%%%%%%%
		  	% git fetch
		  	%%%%%%%%%%%%%%%%%%%%%%%
		  	\rowcolor{tablerowcolor_cyan}
		  	git fetch origin & Lädt alle Änderungen (branches, commits) von dem remote repository origin herunter
		  	\\ \hline


		  	%%%%%%%%%%%%%%%%%%%%%%%
		  	% git init
		  	%%%%%%%%%%%%%%%%%%%%%%%
		  	\rowcolor{tablerowcolor_orange}
		  	git init & Initialisiert ein neues/leeres repository im aktuellen Verzeichnis
		  	\\ \hline


		  	%%%%%%%%%%%%%%%%%%%%%%%
		  	% git log
		  	%%%%%%%%%%%%%%%%%%%%%%%
		  	\rowcolor{tablerowcolor_cyan}
		  	git log & Zeigt alle commits an
		  	\\ \hline
		  	\rowcolor{tablerowcolor_cyan}
		  	git log -n 5 & Zeigt nur die letzten 5 commits an
		  	\\ \hline
		  	\rowcolor{tablerowcolor_cyan}
		  	git log --abbrev-commit & Kürzt die SHA1-Hashes auf 7 Zeichen (oder mehr, wenn 7 Zeichen nicht ausreichen, damit die Hashes unique sind)
		  	\\ \hline
		  	\rowcolor{tablerowcolor_cyan}
		  	git log -p & Zeigt zusätzlich zu den commits alle geänderten Zeilen an
		  	\\ \hline
		  	\rowcolor{tablerowcolor_cyan}
		  	git log -p --word-diff & Zeigt anstatt der geänderten Zeilen nur die geänderten Wörter an
		  	\\ \hline
		  	\rowcolor{tablerowcolor_cyan}
		  	git log --stat & Zeigt zu jedem commit eine kleine Statistik an
		  	\\ \hline


		  	%%%%%%%%%%%%%%%%%%%%%%%
		  	% git merge
		  	%%%%%%%%%%%%%%%%%%%%%%%
		  	\rowcolor{tablerowcolor_orange}
		  	git merge feature & Merged den aktuellen branch mit dem branch feature
		  	\\ \hline


		  	%%%%%%%%%%%%%%%%%%%%%%%
		  	% git mv
		  	%%%%%%%%%%%%%%%%%%%%%%%
		  	\rowcolor{tablerowcolor_cyan}
		  	git mv fileFrom fileTo & Zeigt Datei umbenennen oder verschieden
		  	\\ \hline


		  	%%%%%%%%%%%%%%%%%%%%%%%
		  	% git push
		  	%%%%%%%%%%%%%%%%%%%%%%%
		  	\rowcolor{tablerowcolor_orange}
		  	git push origin master & Lädt das local repository master hoch in das remote repository origin
		  	\\ \hline


		  	%%%%%%%%%%%%%%%%%%%%%%%
		  	% git remote
		  	%%%%%%%%%%%%%%%%%%%%%%%
		  	\rowcolor{tablerowcolor_cyan}
		  	git remote & Zeigt das remote repository an
		  	\\ \hline
		  	\rowcolor{tablerowcolor_cyan}
		  	git remote -v & Zeigt zusätzlich die remote-url an
		  	\\ \hline
		  	\rowcolor{tablerowcolor_cyan}
		  	git remote add [shortname] [url] & Fügt ein neues remote repository hinzu
		  	\\ \hline
		  	\rowcolor{tablerowcolor_cyan}
		  	git remote show origin & Zeigt push/pull-Infos zu dem remote und local repository an
		  	\\ \hline
		  	\rowcolor{tablerowcolor_cyan}
		  	git remote rename oldName newName & Benennt das remote repository oldName um in newName
		  	\\ \hline
		  	\rowcolor{tablerowcolor_cyan}
		  	git remote rm repositoryName & Entfert das remote repository mit der Bezeichnung repositoryName
		  	\\ \hline


		  	%%%%%%%%%%%%%%%%%%%%%%%
		  	% git reset
		  	%%%%%%%%%%%%%%%%%%%%%%%
		  	\rowcolor{tablerowcolor_orange}
		  	git reset HEAD file.txt & Unstaged file.txt
		  	\\ \hline


		  	%%%%%%%%%%%%%%%%%%%%%%%
		  	% git rm
		  	%%%%%%%%%%%%%%%%%%%%%%%
		  	\rowcolor{tablerowcolor_cyan}
		  	git rm file.txt & Merkt auf der stage die Löschung von file.txt für das nächste commit vor und löscht file.txt aus dem Verzeichnis
		  	\\ \hline
		  	\rowcolor{tablerowcolor_cyan}
		  	git rm --cached file.txt & Löscht die Datei nicht aus dem Arbeitsverzeichnis, aber aus der Versionskontrolle
		  	\\ \hline


			%%%%%%%%%%%%%%%%%%%%%%%
		  	% git stats
		  	%%%%%%%%%%%%%%%%%%%%%%%
		  	\rowcolor{tablerowcolor_orange}
		  	git status & Zeigt den aktuellen branch, untracked files, die stage, usw. an
		  	\\ \hline		  	


		\end{longtable}
